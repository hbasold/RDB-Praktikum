\documentclass[a4paper]{article}
\usepackage{a4wide}
\usepackage[utf8]{inputenc}
\usepackage[ngerman]{babel}
\usepackage{amsmath}
\usepackage{graphicx}
\usepackage{listings}

\title{Praktikum „Integritätsbedingungen“, Phase 3}
\author{Henning Basold \and Igor Zerr}
\date{\today}
\begin{document}
\maketitle

\section{Generierte Funktionen und Trigger}

\subsection{HOUSE\_HOUSE\_DISJOINT}
\begin{lstlisting}[language=sql]
CREATE ASSERTION HOUSE_HOUSE_DISJOINT CHECK (
    NOT EXISTS ( SELECT * FROM
        Haus AS h1, Haus AS h2
        WHERE h1.umriss && h2.umriss
    )
);
\end{lstlisting}

\begin{lstlisting}[language=sql]
CREATE FUNCTION check_house_house_disjoint() RETURNS trigger
    LANGUAGE plpgsql
    AS $$Declare res RECORD;
BEGIN
  SELECT INTO res COUNT(*) AS num
  FROM TestSysRel
  WHERE NOT (
    NOT EXISTS ( SELECT * FROM
        Haus AS h1, Haus AS h2
        WHERE h1.umriss && h2.umriss
    )
  )
;  IF (res.num > 0)
  THEN RAISE EXCEPTION
    'ASSERTION CHECK_HOUSE_HOUSE_DISJOINT violated!';
  END IF;
  RETURN NEW;
END;$$;

CREATE TRIGGER check_house_house_disjoint_haus
    AFTER INSERT OR DELETE OR UPDATE ON haus
    FOR EACH ROW
    EXECUTE PROCEDURE check_house_house_disjoint();

CREATE TRIGGER check_house_lake_disjoint_haus
    AFTER INSERT OR DELETE OR UPDATE ON haus
    FOR EACH ROW
    EXECUTE PROCEDURE check_house_lake_disjoint();

\end{lstlisting}

\subsection{NO\_STANDALONE\_STOP}
\begin{lstlisting}[language=sql]
CREATE ASSERTION NO_STANDALONE_STOP CHECK (
    NOT EXISTS ( SELECT * FROM
        Haltestelle AS b
        WHERE NOT EXISTS ( SELECT * FROM
            Strasse AS s
            WHERE approx_circle(b.position, 1) ?# s.verlauf
        )
    )
);
\end{lstlisting}

\begin{lstlisting}[language=sql]
CREATE FUNCTION check_no_standalone_stop() RETURNS trigger
    LANGUAGE plpgsql
    AS $$Declare res RECORD;
BEGIN
  SELECT INTO res COUNT(*) AS num
  FROM TestSysRel
  WHERE NOT (
    NOT EXISTS ( SELECT * FROM
        Haltestelle AS b
        WHERE NOT EXISTS ( SELECT * FROM
            Strasse AS s
            WHERE approx_circle(b.position, 1) ?# s.verlauf
        )
    )
  )
;  IF (res.num > 0)
  THEN RAISE EXCEPTION
    'ASSERTION CHECK_NO_STANDALONE_STOP violated!';
  END IF;
  RETURN NEW;
END;$$;

CREATE TRIGGER check_no_standalone_stop_haltestelle
    AFTER INSERT OR DELETE OR UPDATE ON haltestelle
    FOR EACH ROW
    EXECUTE PROCEDURE check_no_standalone_stop();

CREATE TRIGGER check_no_standalone_stop_strasse
    AFTER INSERT OR DELETE OR UPDATE ON strasse
    FOR EACH ROW
    EXECUTE PROCEDURE check_no_standalone_stop();

\end{lstlisting}

\section{Datenimport}

\section{Bestimmen der betroffenen Relationen}

\begin{lstlisting}[language=java]
Set<String>
getAffectedTables(Connection sql, Assertion a) throws SQLException {
    Pattern tableExtraction
        = Pattern.compile(".* Scan.* on (\\w+) .*");
    
    Set<String> affectedTables
        = new CopyOnWriteArraySet<String>();

    Statement check = sql.createStatement();
    ResultSet pred
        = check.executeQuery(
            "EXPLAIN SELECT * FROM TestSysRel WHERE " + a.predicate);                
    while(pred.next()){
        Matcher match
            = tableExtraction.matcher(pred.getString("QUERY PLAN"));
        if(match.find()){
            String table = match.group(1);
            if(!table.equalsIgnoreCase("TestSysRel")){
                affectedTables.add(table);
            }
        }
    }
    
    return affectedTables;
}
\end{lstlisting}

\end{document}
