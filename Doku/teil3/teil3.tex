\documentclass[a4paper]{article}
\usepackage{a4wide}
\usepackage[utf8]{inputenc}
\usepackage[ngerman]{babel}
\usepackage{amsmath}
\usepackage{graphicx}
\usepackage{listings}

\title{Praktikum „Integritätsbedingungen“, Phase 3}
\author{Henning Basold \and Igor Zerr}
\date{\today}
\begin{document}
\maketitle

\section{Generierte Funktionen und Trigger}

\subsection{HOUSE\_HOUSE\_DISJOINT}
\begin{lstlisting}[language=sql]
CREATE ASSERTION HOUSE_HOUSE_DISJOINT CHECK (
    NOT EXISTS ( SELECT * FROM
        Haus AS h1, Haus AS h2
        WHERE h1.id <> h2.id and h1.umriss && h2.umriss
    )
);
\end{lstlisting}

\begin{lstlisting}[language=sql]
CREATE FUNCTION check_house_house_disjoint() RETURNS trigger
    LANGUAGE plpgsql
    AS $$Declare res RECORD;
BEGIN
  SELECT INTO res COUNT(*) AS num
  FROM TestSysRel
  WHERE NOT (
    NOT EXISTS ( SELECT * FROM
        Haus AS h1, Haus AS h2
        WHERE h1.id <> h2.id and h1.umriss && h2.umriss
    )
  )
;  IF (res.num > 0)
  THEN RAISE EXCEPTION
    'ASSERTION CHECK_HOUSE_HOUSE_DISJOINT violated!';
  END IF;
  RETURN NEW;
END;$$;

CREATE TRIGGER check_house_house_disjoint_haus
    AFTER INSERT OR DELETE OR UPDATE ON haus
    FOR EACH ROW
    EXECUTE PROCEDURE check_house_house_disjoint();

CREATE TRIGGER check_house_lake_disjoint_haus
    AFTER INSERT OR DELETE OR UPDATE ON haus
    FOR EACH ROW
    EXECUTE PROCEDURE check_house_lake_disjoint();

\end{lstlisting}

\subsection{NO\_STANDALONE\_STOP}
\begin{lstlisting}[language=sql]
CREATE ASSERTION NO_STANDALONE_STOP CHECK (
    NOT EXISTS ( SELECT * FROM
        Haltestelle AS b
        WHERE NOT EXISTS ( SELECT * FROM
            Strasse AS s
            WHERE approx_circle(b.position, 1) ?# s.verlauf
        )
    )
);
\end{lstlisting}

\begin{lstlisting}[language=sql]
CREATE FUNCTION check_no_standalone_stop() RETURNS trigger
    LANGUAGE plpgsql
    AS $$Declare res RECORD;
BEGIN
  SELECT INTO res COUNT(*) AS num
  FROM TestSysRel
  WHERE NOT (
    NOT EXISTS ( SELECT * FROM
        Haltestelle AS b
        WHERE NOT EXISTS ( SELECT * FROM
            Strasse AS s
            WHERE approx_circle(b.position, 1) ?# s.verlauf
        )
    )
  )
;  IF (res.num > 0)
  THEN RAISE EXCEPTION
    'ASSERTION CHECK_NO_STANDALONE_STOP violated!';
  END IF;
  RETURN NEW;
END;$$;

CREATE TRIGGER check_no_standalone_stop_haltestelle
    AFTER INSERT OR DELETE OR UPDATE ON haltestelle
    FOR EACH ROW
    EXECUTE PROCEDURE check_no_standalone_stop();

CREATE TRIGGER check_no_standalone_stop_strasse
    AFTER INSERT OR DELETE OR UPDATE ON strasse
    FOR EACH ROW
    EXECUTE PROCEDURE check_no_standalone_stop();

\end{lstlisting}

\section{Datenimport}
\subsection{Übersicht}

\begin{verbatim}
Inserted Tuples:
    Tunnel: 48
    Park: 30
    Strassenbahn: 55
    Spielplatz: 24
    Strasse: 1150
    Fluss: 7
    Parkplatz: 22
    Eisenbahn: 55
    See: 5
    Bruecke: 47
    Haus: 2205
    Landnutzung: 70
Total: 3718

Violated Assertions:
    CHECK_STREET_PLAYGROUND_DISJOINT: 1
    CHECK_PARKING_REACHABLE: 43
    CHECK_STREET_LANDUSE_DISJOINT: 74
    CHECK_NO_STANDALONE_STOP: 167
    CHECK_HOUSE_STREET_DISJOINT: 11
    CHECK_TRAFFIC_LIGHT_AT_STREET: 223
    CHECK_HOUSE_HOUSE_DISJOINT: 1604
    CHECK_HOUSE_LAKE_DISJOINT: 0
\end{verbatim}

\subsection{Nicht verletzte Assertions}

\subsubsection{HOUSE\_STREET\_DISJOINT}
SQL Verletzung: 
\begin{lstlisting}[language=sql,breaklines=true]
INSERT INTO Haus (id,umriss, hausnummer, strasse, ort, plz, nutzung) values('3457842','((52.2658021,10.5088862),(52.2658321,10.5082349))','42','Kasernenstrasse','Braunschweig','38102',NULL)
\end{lstlisting}

\subsubsection{HOUSE\_LAKE\_DISJOINT}
SQL Verletzung:
\begin{lstlisting}[language=sql,breaklines=true]
INSERT INTO Haus (id,umriss, hausnummer, strasse, ort, plz, nutzung) values('3457843','((52.2593691,10.530547),(52.2593475,10.5305934),(52.2593108,10.530598),(52.2592873,10.5305653),(52.2592827,10.5305139),(52.2592984,10.5304732),(52.2593168,10.5304591),(52.2593452,10.5304646),(52.2593669,10.5305019),(52.2593691,10.530547))','42','Kasernenstrasse','Braunschweig','38102',NULL)
\end{lstlisting}

\subsubsection{STREET\_LAKE\_DISJOINT}
SQL Verletzung: 
\begin{lstlisting}[language=sql,breaklines=true]
INSERT INTO Strasse (id,verlauf,strassentyp,name,maxV) values('23251939','[(52.2601605,10.5313293),(52.2601389,10.5313757),(52.2601022,10.5313803),(52.2600787,10.5313476),(52.2600742,10.5312962),(52.2600898,10.5312555),(52.2601082,10.5312414),(52.2601366,10.5312469),(52.2601583,10.5312842),(52.2601605,10.5313293)]','residential','Neue Knochenhauerstrasse','0')
\end{lstlisting}

\subsubsection{WATER\_CROSSING}
\begin{lstlisting}[language=sql,breaklines=true]
INSERT INTO Bruecke (id,verlauf,name,maxV) values('3457844','[(52.2674258,10.5246387),(52.2669683,10.5250233)]','Leonhardstrasse','30')
\end{lstlisting}

\section{Bestimmen der betroffenen Relationen}

\begin{lstlisting}[language=java]
Set<String>
getAffectedTables(Connection sql, Assertion a) throws SQLException {
    Pattern tableExtraction
        = Pattern.compile(".* Scan.* on (\\w+) .*");
    
    Set<String> affectedTables
        = new CopyOnWriteArraySet<String>();

    Statement check = sql.createStatement();
    ResultSet pred
        = check.executeQuery(
            "EXPLAIN SELECT * FROM TestSysRel WHERE " + a.predicate);                
    while(pred.next()){
        Matcher match
            = tableExtraction.matcher(pred.getString("QUERY PLAN"));
        if(match.find()){
            String table = match.group(1);
            if(!table.equalsIgnoreCase("TestSysRel")){
                affectedTables.add(table);
            }
        }
    }
    
    return affectedTables;
}
\end{lstlisting}

\end{document}
