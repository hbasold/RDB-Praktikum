\documentclass[a4paper]{article}
\usepackage{a4wide}
\usepackage[utf8]{inputenc}
\usepackage[ngerman]{babel}
\usepackage{amsmath}
\usepackage{graphicx}
\usepackage{listings}

\title{Praktikum „Integritätsbedingungen“, Phase 2}
\author{Henning Basold \and Igor Zerr}
\date{\today}
\begin{document}
\maketitle

\section{Fehlerhafte Assertions}

\subsection{Fehlendes Schlüsselwort „Assertion“}
\lstinputlisting[language=SQL]{assertions_missing_keyword.asn}

\begin{verbatim}
Parse error: Parse error at 1:8:
     Expected token "ASSERTION" but got "H"
\end{verbatim}

\subsection{Fehlendes Semikolon}
\lstinputlisting[language=SQL]{assertions_missing_semicolon.asn}

\begin{verbatim}
Parse error: Parse error at 8:1:
    Expected ";"
\end{verbatim}

\subsection{Fehlende schließende Klammer}
\lstinputlisting[language=SQL]{assertions_missing_close_paren.asn}

\begin{verbatim}
Error in assertion HOUSE_STREET_DISJOINT on line 1: 
    error in predicate at position 132: Syntaxfehler am Ende der Eingabe
\end{verbatim}

\begin{verbatim}
\end{verbatim}

\subsection{Syntaktisch inkorrekter Bezeichner}
\lstinputlisting[language=SQL]{assertions_invalid_identifier.asn}

\begin{verbatim}
Parse error: Parse error at 1:18:
    Expected identifier but got "1234"
\end{verbatim}

\subsection{Schlüsselwort als Bezeichner}
\lstinputlisting[language=SQL]{assertions_keyword_identifier.asn}

\begin{verbatim}
Error in assertion select on line 1: 
    "select" is an invalid SQL identifier
\end{verbatim}

\subsection{Syntaktisch ungültiges Prädikat}
\lstinputlisting[language=SQL]{assertions_invalid_predicate.asn}

\begin{verbatim}
Error in assertion HOUSE_STREET_DISJOINT on line 1: 
    syntactic error in predicate at position 32 near "CREATE".
\end{verbatim}

\subsection{Fehlende Tabelle in Prädikat}
\lstinputlisting[language=SQL]{assertions_missing_table.asn}

\begin{verbatim}
Error in assertion HOUSE_STREET_DISJOINT on line 1: 
    error in predicate at position 67: Relation "xyz" does not exist.
\end{verbatim}

\subsection{Fehlende Spalte in Prädikat}
\lstinputlisting[language=SQL]{assertions_missing_column.asn}

\begin{verbatim}
Error in assertion HOUSE_STREET_DISJOINT on line 1: 
    error in predicate at position 110: Spalte "h.xyz" does not exist.
\end{verbatim}

\subsection{Ungültiger Operator in Prädikat}
\lstinputlisting[language=SQL]{assertions_invalid_operator.asn}

\begin{verbatim}
Error in assertion invPred on line 1: 
    error in predicate at position 120: Operator existiert nicht: path ?+ path
  Hinweis: Kein Operator stimmt mit dem angegebenen Namen und den Argumenttypen überein.
  Sie müssen möglicherweise ausdrückliche Typumwandlungen hinzufügen.
\end{verbatim}

\subsection{Mehrfaches Einfügen}

Die erste Assertion ist bereits in der Datenbank eingetragen, hat
aber das gleiche Prädikat. Die zweite hat ein anderes Prädikat, aber den gleichen
Namen.
\lstinputlisting[language=SQL]{assertions_dual_insert.asn}

\begin{verbatim}
Warning: assertion HOUSE_STREET_DISJOINT on line 1 already exists with same predicate.
    It is not inserted again.
Error in assertion HOUSE_STREET_DISJOINT on line 8: 
    assertion already exists with another predicate.
\end{verbatim}

\section{Korrekte Assertions}
\lstinputlisting[language=SQL]{assertions.asn}

\begin{verbatim}
Warning: assertion HOUSE_STREET_DISJOINT on line 1 already exists with same predicate.
    It is not inserted again.
Warning: assertion HOUSE_HOUSE_DISJOINT on line 8 already exists with same predicate.
    It is not inserted again.
Warning: assertion HOUSE_LAKE_DISJOINT on line 15 already exists with same predicate.
    It is not inserted again.
Warning: assertion STREET_LAKE_DISJOINT on line 22 already exists with same predicate.
    It is not inserted again.
Warning: assertion STREET_LANDUSE_DISJOINT on line 29 already exists with same predicate.
    It is not inserted again.
Warning: assertion STREET_PLAYGROUND_DISJOINT on line 36 already exists with same predicate.
    It is not inserted again.
Warning: assertion NO_STANDALONE_STOP on line 43 already exists with same predicate.
    It is not inserted again.
Warning: assertion TRAFFIC_LIGHT_AT_STREET on line 53 already exists with same predicate.
    It is not inserted again.
Warning: assertion PARKING_REACHABLE on line 63 already exists with same predicate.
    It is not inserted again.
Assertions have been successfully checked and saved.
\end{verbatim}

Datenbankinhalt:

\begin{lstlisting}[breaklines=true]
pg_dump -a -t AssertionSysRel rdb_praktikum

--
-- PostgreSQL database dump
--

SET statement_timeout = 0;
SET client_encoding = 'UTF8';
SET standard_conforming_strings = off;
SET check_function_bodies = false;
SET client_min_messages = warning;
SET escape_string_warning = off;

SET search_path = public, pg_catalog;

--
-- Data for Name: assertionsysrel; Type: TABLE DATA; Schema: public; Owner: ***
--

COPY assertionsysrel (assertionname, bedingung, implementiert) FROM stdin;
HOUSE_STREET_DISJOINT	NOT EXISTS ( SELECT * FROM\n        Haus AS h, Strasse AS s\n        WHERE path(h.umriss) ?# s.verlauf\n    )	f
HOUSE_HOUSE_DISJOINT	NOT EXISTS ( SELECT * FROM\n        Haus AS h1, Haus AS h2\n        WHERE h1.umriss && h2.umriss\n    )	f
HOUSE_LAKE_DISJOINT	NOT EXISTS ( SELECT * FROM\n        Haus AS h, See AS l\n        WHERE h.umriss && l.umriss\n    )	f
STREET_LAKE_DISJOINT	NOT EXISTS ( SELECT * FROM\n        Strasse AS s, See AS l\n        WHERE s.verlauf ?# path(l.umriss)\n    )	f
STREET_LANDUSE_DISJOINT	NOT EXISTS ( SELECT * FROM\n        Strasse AS s, Landnutzung AS l\n        WHERE s.verlauf ?# path(l.umriss)\n    )	f
STREET_PLAYGROUND_DISJOINT	NOT EXISTS ( SELECT * FROM\n        Strasse AS s, Spielplatz AS p\n        WHERE s.verlauf ?# path(p.umriss)\n    )	f
NO_STANDALONE_STOP	NOT EXISTS ( SELECT * FROM\n        Haltestelle AS b\n        WHERE NOT EXISTS ( SELECT * FROM\n            Strasse AS s\n            WHERE approx_circle(b.position, 1) ?# s.verlauf\n        )\n    )f
TRAFFIC_LIGHT_AT_STREET	NOT EXISTS ( SELECT * FROM\n        Ampel AS t\n        WHERE NOT EXISTS ( SELECT * FROM\n            Strasse AS s\n            WHERE approx_circle(t.position, 1) ?# s.verlauf\n        )\n    )	f
PARKING_REACHABLE	NOT EXISTS ( SELECT * FROM\n        Parkplatz AS p\n        WHERE NOT EXISTS ( SELECT * FROM\n            Strasse AS s\n            WHERE approx_circle(p.position, 1) ?# s.verlauf\n        )\n    )	f
\.


--
-- PostgreSQL database dump complete
--
\end{lstlisting}

\section{Prüfung der Bezeichner}

\subsection{Prüfung beim Parsen}
\begin{lstlisting}[language=java,mathescape=true]

identifierPattern = Pattern.compile("[_a-zA-Z][_a-zA-Z0-9]*");

$\vdots$

private String
parseIndentifier(InputStreamIterator in) throws AssertionParseError {
    int oldLine = line;
    int oldColumn = column;

    StringBuffer word = new StringBuffer();

    if(!isWS(next)){
        word.appendCodePoint(next.intValue());
    }

    while(in.hasNext()) {
        nextChar(in);
        if(!isWS(next)){
            word.appendCodePoint(next.intValue());
        }
        else{
            break;
        }
    }

    Matcher idMatcher = identifierPattern.matcher(word);
    if(idMatcher.matches()){
        return word.toString();
    }
    else{
        throw new AssertionParseError("Expected identifier but got \""
                                        + word + "\"", oldLine, oldColumn);
    }
}
\end{lstlisting}

\subsection{Prüfung auf Verträglichkeit mit SQL}
\begin{lstlisting}[language=java,mathescape=true,escapechar=\%]

syntaxErrorParser
    = Pattern.compile(".* Syntaxfehler bei %\frqq%(.*)%\flqq%\\s+ Position: (\\d+).*");

$\vdots$

private String checkName(String name) throws SQLException {
    Statement create = sql.createStatement();
    try {
        create.executeUpdate("CREATE TABLE " + name + " (Attribut INTEGER)");
        create.executeUpdate("DROP TABLE " + name);
    }
    catch (SQLException e) {
        // Leider funktionieren die Fehlercodes etc.
        // mit dem Postgres-Backend scheinbar nicht...
        Matcher m = syntaxErrorParser.matcher(e.getMessage());
        if(m.matches()){
            return "\"" + name + "\" is an invalid SQL identifier";
        }
        else{
            throw e;
        }
    }
    finally {
        create.close();
    }

    return null;
}
\end{lstlisting}

\end{document}
